\documentclass{sig-alternate-05-2015}

\begin{document}


\title{Visualizacion de Grafos}
\subtitle{Uso de SNAP para procesar grafos y Graphy para visualizacion}


% TODO: Update address.
% TODO: Check university length. 
\author{
Gerardo Garcia Teruel Noriega\\
       \affaddr{Instituto Tecnologico y de Estudios Superiores de Monterrey, Campus Santa Fe}\\
       \affaddr{1932 Wallamaloo Lane}\\
       \affaddr{Mexico DF, Mexico}\\
       \email{A01018057@itesm.mx}
       }
       
\maketitle


% TODO: Update abstract. 
\begin{abstract}
This paper provides a sample of a \LaTeX\ document which conforms,
somewhat loosely, to the formatting guidelines for
ACM SIG Proceedings. It is an {\em alternate} style which produces
a {\em tighter-looking} paper and was designed in response to
concerns expressed, by authors, over page-budgets.
It complements the document \textit{Author's (Alternate) Guide to
Preparing ACM SIG Proceedings Using \LaTeX$2_\epsilon$\ and Bib\TeX}.
This source file has been written with the intention of being
compiled under \LaTeX$2_\epsilon$\ and BibTeX.

The developers have tried to include every imaginable sort
of ``bells and whistles", such as a subtitle, footnotes on
title, subtitle and authors, as well as in the text, and
every optional component (e.g. Acknowledgments, Additional
Authors, Appendices), not to mention examples of
equations, theorems, tables and figures.

To make best use of this sample document, run it through \LaTeX\
and BibTeX, and compare this source code with the printed
output produced by the dvi file. A compiled PDF version
is available on the web page to help you with the
`look and feel'.
\end{abstract}



\keywords{Graph, Snap library, Graphy, Twitter, Network Analysis, Graph Visualization; \LaTeX; text tagging}


\section{Introduccion}
En este art�culo, se genera un grafo de conexiones en Twitter utilizando un dataset de SNAP y proces�ndolo en C++ con la misma librer�a de SNAP. El procesamiento del grafo mediante la exportaci�n en 5 formatos: GraphML, GEXF, GDF y GraphSon. Posteriormente se utilizo Gephi para visualizar el grafo y obtener informaci�n relevante para su an�lisis. 

\section{Procesmiento del grafo}

\subsection{Importacion del grafo a SNAP}
Se descargo la informaci'on del grafo de https://snap.stanford.edu/data/index.html, el formato se descomprimio en formato txt y este a su vez fue importado a snap utilizando la siguiente linea de codigo: 

\begin{verbatim}
PGraph Graph = TSnap::LoadEdgeList<PUNGraph>("twitter_combined.txt",0,1);
\end{verbatim}

\subsection{Exportacion del grafo}
El grafo se exporto en 5 formatos: GraphML, GEXF, GDF y GraphSON. 

\subsubsection{GraphML}
GraphML es un formato para almacenar grafos basado en XML disenado bajo los pilares de: simplicidad, generalidad, extensibilidad, robusto. El formato consiste en 



Busque en Internet la biblioteca para la implementaci�n de grafos SNAP y genere con la misma un grafo a partir de alguno de los datasets que aparecen en https://snap.stanford.edu/data/index.html. Una vez importado el dataset seleccionado  implemente las funciones que permitan:

Exportar el grafo en formato GraphML
Exportar el grafo en formato GEXF
Exportar el grafo en formato GDF
Exportar el grafo en formato JSON Graph Format (vea GraphSON)
Posteriormente, exporte el grafo en cada uno de los formatos especificados y utilizando un programa como Gephi, importe uno de los grafos y muestre su representaci�n gr�fica. Explore las opciones de la herramienta Gephi y obtenga informaci�n relevante referente a la informaci�n almacenada en el grafo.
Finalmente, genere un art�culo de investigaci�n donde incluya:
la actividad realizada paso a paso, incluyendo im�genes del grafo una vez visualizado en Gephi
las ventajas y desventajas de cada formato de exportaci�n utilizado, 
la complejidad temporal y espacial de cada una de las funciones de exportaci�n (4 en total)
el tiempo de ejecuci�n de cada funci�n de exportaci�n
las ventajas y desventajas que tiene la utilizaci�n de programas como Gephi para la visualizaci�n de grafos
las referencias bibliogr�ficas en el formato correcto
los c�digos de cada funci�n de exportaci�n como anexos del art�culo
la liga al repositorio de GitHub donde se encuentren los c�digos programados
El articulo de investigaci�n debe seguir el formato de la ACM el cual puede encontrar en http://www.acm.org/publications/article-templates/proceedings-template.html/ . La fecha de t�rmino de la actividad es el 29 de octubre de 2015 a las 23:55.
No se aceptan trabajos fuera de fecha ni por correo electr�nico. En ambos casos la calificaci�n de la tarea ser� 0 puntos.

\end{document}